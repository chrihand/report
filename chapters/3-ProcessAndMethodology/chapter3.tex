%===================================== CHAP 3 =================================

\chapter{Process and Methodology}

\section{Methodology}
\label{methodology}
During the development of the product the group decided to utilise scrum and the agile methodology. This was a natural choice of methodology as the customer required the process to be iterative and based on the group's previous experiences.\\ 

\noindent The process methodology was adjusted to suit the project in the best possible manner, as well as to suit the group's needs. The group completed sprint reviews, sprint retrospectives and daily meetings to achieve the best possible outcome of the project. The first 15 minutes of the daily meetings was used to set an agenda and goal of the day, and was completed by using the last 15 minutes to summarize the day. The product owner of the project is our customer, and will only be referred to as the customer throughout this document. To maintain consistent communication with the customer, there were scheduled stand-up meetings bi-weekly or if needed. These meetings allowed the group to continuously update the customer and demonstrate prototypes of the product. This ensured that the product developed corresponded to customer's vision. To aim for a product covering the end users needs, the customer provided a set of test users, to test the product iteratively during the development. \\

\noindent During the development process the group also utilised the principles of Extreme Programming, specifically the practice of pair programming. A decision based upon the groups desire to maximize the learning opportunities from each other resulting in the best possible outcome. 

\subsection{Development}
Throughout the development the group utilised the practice of issue tracking on GitHub (see chapter  \ref{GitHub}), and structured the tasks based on a roadmap using milestones. To ensure that the group followed the principles of scrum - it was decided to define the milestones as a correspondence to time, with each sprint as a milestone in the project. \\

\noindent To easily track the issues during the development the group used waffle.io (see chapter  \ref{Waffle.io}) as a scrum board. This provided the group a visualisation of the process - with issues in 'product backlog', 'sprint backlog' and 'in process' including who is working on which using assignees. 

\subsection{Report}
Waffle.io was also used during the process of writing the report - primarily to ensure that every member participates in writing the report, and every member proof-reads each part before it is categorized as done.  


\section{Project Organization}\label{projectOrganisation}
The group decided to divide the project into reasonable areas - and delegate the main responsibility to two of the group members, where one had the primary responsibility. This would ensure that all parts of the project were kept in mind during the development, as well as more than one person knew what was going on in one specific area.\\
Areas of main responsibility is presented in bold.

\subsection{Project Owner}
\textbf{Babak Farshchian}

\subsection{Scrum Master}
\textbf{Ingrid Skar} - \emph{\textbf{Design/GUI}, Front-end}

\subsection{Scrum Team}
Roles denoted in \textbf{bold}, represent that a person has main responsibility for that subject.
\textbf{Christina Hellenes Andresen} - \emph{\textbf{Group leader}, \textbf{Report} and Design/GUI}\\
\noindent \textbf{Andreas Norstein} - \emph{\textbf{Architect}, \textbf{Database} and Back-end}\\
\noindent \textbf{Sigve Andre Evensen Skaugvoll} - \emph{\textbf{Contact responsible with customer and supervisor}, \textbf{Back-end}, \textbf{Database}, \textbf{Tester}, \textbf{Elected Representative}, Architect and Integration}\\
\noindent \textbf{Martin Stigen} - \emph{\textbf{Integration}, Tester and Front-end} \\
\noindent \textbf{Thomas Wold} - \emph{\textbf{Front-end}, Report and Back-end}\\ 

\section{Sprints and Milestones}
\label{sprintsAndMilestones}
The group divided the development process into sprints, were each sprint lasted for two weeks running Monday to Friday, and planned with respect to the given milestones. 

\subsection{Sprints}
\begin{longtable}{@{\extracolsep{\fill}}
                |L{0.16\linewidth}
                |L{0.18\linewidth}
                |L{0.12\linewidth}
                |L{0.40\linewidth}|@{}}
\hline
\rowcolor{Gray}
\textbf{Dates}&\textbf{Planned Work Hours}&\textbf{Sprint}&\textbf{Goal}\\
\hline
06.02 - 10.02&136.5&Sprint 0&Initialization phase. Planning, product backlog set up and first draft design.\\
\hline
13.02 - 24.02&270&Sprint 1& Deliver first interactive prototype to customer.\\
\hline
27.02 - 10.03&270&Sprint 2&-\\
\hline
13.03 - 24.03&270&Sprint 3&-\\
\hline
27.03 - 07.04&270&Sprint 4&-\\
\hline
18.04 - 28.04&270&Sprint 5&Find and fix bugs.\\
\hline
01.05 - 12.05&270&Sprint 6&Final touch report\\
\hline
15.05 - 26.05&270&Sprint 7&First aid kit\\
\hline
27.05 - 30.06&270&Sprint 8&Deliver\\
\hline
\caption{Sprints}
\end{longtable}

\subsection{Milestones}
The milestones are the important dates given in the project and an internal deadline. In addition to this, the milestones consist of the project's phases - in accordance to GitHub's guidelines \cite{GitHubGuide}. 

\begin{longtable}{|l|l|}
\hline
\rowcolor{Gray}
\textbf{Dates} & \textbf{Milestone} \\
\hline
15.02 & Very preliminary delivery of report \\
\hline
19.02 & Mid semester version of report\\
\hline
07.04 & Group deadline. No new features should be implemented. \\
\hline
08.05 - 12.05 & Final Presentation \\
\hline
30.05 & Deliver  \\
\hline
\caption{Milestones}
\end{longtable}


\subsection{Workbreakdown}


\subsection{Gantt Diagram} 

\cleardoublepage