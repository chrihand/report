%===================================== CHAP 4 =================================

\chapter{Requirements}
\label{requirements}
This chapter describes the requirements for the product. Consisting of a definition of the Minimum Viable Product, the functional- and non-functional requirements, and changes made to the requirements.

\section{Minimum Viable Product}
\label{MVP}
The group represented their commitment and shared goal by specifying a definition of the Minimum Viable Product (denoted as MVP), which had to be delivered when the project was due. The first MVP was defined during sprint 0 (see section \ref{sprint0}), and was further defined as the project developed. See Appendix \ref{original_minimal_viable_product} for the first MVP, the final definition is presented below. 

\textit{The MVP the group is to deliver is a web portal with a homepage, an overall activity page, an overall provider page, create activity page, and profile pages. The web portal's homepage presents upcoming events to the user. The portal is a user account service; meaning the information on the homepage depends on the user logged in, such as 'my attending activities' if the user are attending any. Users are allowed to log in using an Facebook account, such that no one are required to create accounts to use the portal. The portal will have pages displaying all registered activities and providers, which allows for search and filtering. The portal allows providers to create activities, either from scratch or based on Facebook events, which are presented to all users.}

\textit{The design of the portal, will be under the WCAG 2.0 protocol, which state rules for universal design.}

\section{Functional Requirements}
\label{functional_requirements}
This section contains the functional requirements for the product, which specifies the behavior of the system \cite{requirements}. These were made by the group, based on experiences from the workshop with the users and customer input. The requirements were further verified by the customer, who also added some he thought were important. 

These are as follows:
\begin{itemize}[noitemsep]
    \item Log in.
    \item See complete list of activities, and filter these.
    \item See complete list of providers, and filter these.
    \item Follow Providers. 
    \item Administer Activities.
    \item See all information about one activity.
    \item Give feedback on an activity.
\end{itemize}


\subsection{User Stories}
\label{User stories}
The group created user stories (see tables \ref{User_Stories_login} to \ref{User_Stories_General}) based on the functional requirements. These were then verified by the customer, who also added priorities to guide the group. The user stories contain ID's to identify each one. Since the importance of different user functionalities had been especially emphasized during the workshop, the group decided to separate the user stories based on different user groups. This included the general user (US.xx), children(CH.xx), parents(PA.xx), providers(PR.xx) and system maintainers(SM.xx). The acceptance criteria are the criteria which have to be fulfilled for the user story to be categorized as done. The group also defined definitions of when a user story is ready and of when a user story is done, see Appendix \ref{user_story_definition}. The group actively used these definitions during the development; to ensure the group that what is being created was a necessity, and when to declare it as complete and proceed to the next user story.  

The user stories are all listed below, in respect to the functional requirements. 

\subsubsection{Log in}
\begin{longtable}{@{\extracolsep{\fill}}
                |L{0.10\linewidth}
                |L{0.33\linewidth}
                |L{0.33\linewidth}
                |L{0.10\linewidth}|@{}}
\hline
\rowcolor{Gray}
\textbf{ID} & \textbf{User Story} & \textbf{Acceptance criteria} & \textbf{Priority} \\
\hline
US.03 & \textbf{As a} user, \textbf{I want} to be able to log in, \textbf{so that I can} sign up for events and see my personal page. & A user must be able to log in. Get feedback after logged in. Stay logged in after refreshing page. Be able to log out. & 100 \\
\hline
US.08 & \textbf{As a} user, \textbf{I want} to be able to use Facebook to log in, \textbf{so that I} do not have to register an user to use the page. & A user must be able to log in using an Facebook account. Get feedback after logged in. Stay logged in after refreshing page. Be able to log out. & 100 \\
\hline
\caption{User Stories - Log In}
\label{User_Stories_login}
\end{longtable}

\subsubsection{See complete list of activities, and filter these}
\begin{longtable}{@{\extracolsep{\fill}}
                |L{0.10\linewidth}
                |L{0.33\linewidth}
                |L{0.33\linewidth}
                |L{0.10\linewidth}|@{}}
\hline
\rowcolor{Gray}
\textbf{ID} & \textbf{User Story} & \textbf{Acceptance criteria} & \textbf{Priority} \\
\hline
US.00 & \textbf{As a} user, \textbf{I want} to see upcoming activities \textbf{so that I can} get an overview of my opportunities. & The front page includes four upcoming activities which the user can attend, furthermore the All Activities page includes all upcoming activities.& 100\\
\hline
US.04 & \textbf{As a} user, \textbf{I want} a personal page, \textbf{so that I can} see my upcoming activities. & Be able to see personal information. See the activities the user have signed up for. & 70 \\
\hline
US.06 & \textbf{As a} user, \textbf{I want} to see as many activities as possible, not limited to one actor such as the municipality,  \textbf{so that I can} get an full overview of my opportunities. & The user is able to see all upcoming activities on one page. & 100 \\
\hline
CH.00 & \textbf{As a} child, \textbf{I want} see relevant activities for me, \textbf{so that I can} get an overview of my opportunities. & The user is able to search and filter activities based on adaption and activity type & 70 \\
\hline
CH.01 & \textbf{As a} child, \textbf{I want} an overview of activities I am attending, \textbf{so that I can} schedule my life. & When logged in, the Front Page and My Page includes activities the user is attending & 90\\
\hline
CH.03 & \textbf{As a} child, \textbf{I want} to see all activities, \textbf{so that I can} see what activities are available. & The user should be able to view all activities & 100 \\  
\hline
PA.00 & \textbf{As a} parent, \textbf{I want} to see relevant activities for my child, \textbf{so that I} see if the respective activity is suitable for my child. & The user is able to search and filter activities based on adaption and activity type & 100 \\
\hline
PA.01 & \textbf{As a} parent, \textbf{I want} to see all activities for my region in one place, \textbf{so that I} don’t have to search the entire Internet. & The page includes activities that is arranged in Trondheim kommune & 100 \\
\hline
\caption{User Stories - Activities}
\label{User_Stories_Activities}
\end{longtable}

\subsubsection{See complete list of providers, and filter these}
\begin{longtable}{@{\extracolsep{\fill}}
                |L{0.10\linewidth}
                |L{0.33\linewidth}
                |L{0.33\linewidth}
                |L{0.10\linewidth}|@{}}
\hline
\rowcolor{Gray}
\textbf{ID} & \textbf{User Story} & \textbf{Acceptance criteria} & \textbf{Priority} \\
\hline
CH.06 & \textbf{As a} child, \textbf{I want} to see all activities, \textbf{so that I can} see what activities are available. & The user should be able to view all activities & 100 \\  
\hline
PA.08 & \textbf{As a} parent, \textbf{I want} to see relevant providers for my child, \textbf{so that I} see if the respective provider offers activities suitable for my child. & The user is able to search and filter provider based on adaption and activity type & 100 \\
\hline
PA.09 & \textbf{As a} parent, \textbf{I want} to see all providers for my region in one place, \textbf{so that I} don’t have to search the entire Internet. & The page includes providers from Aktørbasen \ref{Aktordatabasen} & 100 \\
\hline
\caption{User Stories - Providers}
\label{User_Stories_Providers}
\end{longtable}

\subsubsection{Follow Providers}
\begin{longtable}{@{\extracolsep{\fill}}
                |L{0.10\linewidth}
                |L{0.33\linewidth}
                |L{0.33\linewidth}
                |L{0.10\linewidth}|@{}}
\hline
\rowcolor{Gray}
\textbf{ID} & \textbf{User Story} & \textbf{Acceptance criteria} & \textbf{Priority} \\
\hline
CH.04 & \textbf{As a} child, \textbf{I want} to be able to follow providers, \textbf{so that I can} get updates when the provider are hosting activities. & The user should be able to follow an provider. & 100 \\  
\hline
CH.05 & \textbf{As a} child, \textbf{I want} to see all activities the provider I am following are hosting, \textbf{so that I} can easily access activities hosted by providers I like. & When the user has followed a provider, it is included in the 'Providers Following' list on My Page. When the user clicks the providers name, it is navigated to a page including all activities hosted by the respective provider.  & 100 \\  
\hline
\caption{User Stories - Follow Provider}
\label{User_Stories_Following}
\end{longtable}


\subsubsection{Administer Activities}
\begin{longtable}{@{\extracolsep{\fill}}
                |L{0.10\linewidth}
                |L{0.33\linewidth}
                |L{0.33\linewidth}
                |L{0.10\linewidth}|@{}}
\hline
\rowcolor{Gray}
\textbf{ID} & \textbf{User Story} & \textbf{Acceptance criteria} & \textbf{Priority} \\
\hline
PA.02 & \textbf{As a} parent, \textbf{I want} to add activities I know about, \textbf{so that I can} help others and to improve the selection of activities & When logged in the parent can locate add activity button, fill in information and save the activity. & 90\\
\hline
PR.00 & \textbf{As a} provider, \textbf{I want} to easily register detailed information regarding facilitation for each activities, \textbf{so that} others can find my activities. & When logged in the provider can locate add activity button, fill in information and save the activity. The activity is shown on All Activities page. & 80\\
\hline
PR.02 & \textbf{As a} provider, \textbf{I want} to easily register my events in one and only one place, \textbf{so that} maintenance and duplication is avoided. & When logged in the provider can locate add activity button, fill in information and save the activity, based on a Facebook activity. The activity is shown on All Activities page. & 100\\
\hline
PR.03 & \textbf{As a} provider, \textbf{I want} the opportunity to add picture / video for an activity,
 \textbf{so that I can} present the activity in a good way. & The provider can add images when creating a new activity, both from own computer and Instagram. & 70 \\
\hline
\caption{User Stories - Administer Activities}
\label{User_Stories_AdminAct}
\end{longtable}

\subsubsection{See all information about one activity}
\begin{longtable}{@{\extracolsep{\fill}}
                |L{0.10\linewidth}
                |L{0.33\linewidth}
                |L{0.33\linewidth}
                |L{0.10\linewidth}|@{}}
\hline
\rowcolor{Gray}
\textbf{ID} & \textbf{User Story} & \textbf{Acceptance criteria} & \textbf{Priority} \\
\hline
US.01 & \textbf{As a} user, \textbf{I want} to see promo (video / pictures) of the activities, \textbf{so that I can} see the purpose of the activity, and what equipment is necessary, etc.. & The user can add a image when creating an activity, and the image is shown on the activity on front page and All Activities page. & 100 \\
\hline
\caption{User Stories - One Activity}
\label{User_Stories_Activity}
\end{longtable}

\subsubsection{Give feedback on an activity}
\begin{longtable}{@{\extracolsep{\fill}}
                |L{0.10\linewidth}
                |L{0.33\linewidth}
                |L{0.33\linewidth}
                |L{0.10\linewidth}|@{}}
\hline
\rowcolor{Gray}
\textbf{ID} & \textbf{User Story} & \textbf{Acceptance criteria} & \textbf{Priority} \\
\hline
PA.03 & \textbf{As a} parent, \textbf{I want} to give constructive feedback on what did/not work, \textbf{so that} other parents can filter out activities. & The user can add a comment in the activity modal. & 70\\
\hline
PA.06 & \textbf{As a} parent, \textbf{I want} have a informal contact channel, \textbf{so that I can} quick and easy ask questions directly to the provider. & The user can add a comment in the activity modal. & 80 \\
\hline
PR.01 & \textbf{As a} provider, \textbf{I want} to see and show ratings on my activities, \textbf{so that I can} improve my activities. & The user can see the rate information on the activity modal. & 80 \\
\hline
\caption{User Stories - Feedback}
\label{User_Stories_Feedback}
\end{longtable}

\subsubsection{General User Stories}
\begin{longtable}{@{\extracolsep{\fill}}
                |L{0.10\linewidth}
                |L{0.33\linewidth}
                |L{0.33\linewidth}
                |L{0.10\linewidth}|@{}}
\hline
\rowcolor{Gray}
\textbf{ID} & \textbf{User Story} & \textbf{Acceptance criteria} & \textbf{Priority} \\
\hline
US.05 & \textbf{As a} user, \textbf{I want} to be able to easily navigate between subpages on the webpage, \textbf{so that I can} browse the webpage efficiently. & Navigation bar should be fully functional. Refreshing the webpage should not change the webpage. & 100 \\
\hline
SM.00 & \textbf{As a} system maintainer, \textbf{I want}  easily install the system in a virtual private server, \textbf{so that} I can easily give access to others. & An installation guide on how to install the system is in appendix \ref{Installation Guide}. & 100 \\
\hline
SM.01 & \textbf{As a} system maintainer, \textbf{I want} to be able to make the system easily available through a standard domain name, \textbf{so that I can} improve my reach. & The web page is accessed on SkalVi.no. & 100 \\
\hline
SM.02 & \textbf{As a} system maintainer, \textbf{I want} to be able to secure my data, \textbf{so that}  users can trust me. & The system uses virtual provider server, therefor the system maintainer is the only one who has access to personal information. & 100\\
\hline
SM.03  & \textbf{As a} system maintainer, \textbf{I want} be able to integrate with new event systems systems, e.g. Facebook, Events Trondheim, \textbf{so that} users can add a variety of events easily. & The user are able to create events based on Facebook. & 100 \\
\hline
\caption{User Stories - General}
\label{User_Stories_General}
\end{longtable}
 
\section{Non-Functional Requirements}
\textit{Non-functional requirements specifies criteria that can be used to judge the operation of a system, rather than specific behaviors} \cite{requirements}. The non-functional requirements were defined by the customer.

\subsection{Open Source}
Open source is defined as the source code to a product is made available to everyone. The code can be published under several licence agreements.

\subsubsection{Apache2 Licence Agreement}
The customer required that the product was published as open source under the licence Apache2. Apache2 is a licence that does not restrict the re-usage of the source code \cite{apache2}. The source code is available to all; it can be reused, further developed, and distributed as long as the required original notice follow the source code and is maintained. 

\subsection{Server Availability}
Server availability became a non-functional requirement in the Sprint 2 (see section \ref{sprint2}), when the customer requested that the latest functional version was available on \url{www.SkalVi.no} at all times. The customer also wanted the server to be updated at least once a week, following the principles of the chosen methodology (see section \ref{methodology}). The group decided that the server would be updated two times a week, Wednesday and Fridays. As the group worked iteratively, according to the chosen development methodology, this request was incorporated in the current sprint. Furthermore, this revealed that changes to the architecture was required to avoid potential big problems; such as production- and development configuration, and class dependencies.

\subsection{Security}
As the web portal allowed users to register and log in, handling personal information in a good manner was considered important. By using the user authentication system provided by the Django framework both user access and privacy were achieved. Allowing user to use Facebook accounts to log in was also a choice based on security, since Facebook handles secure log in.


\subsection{Universal Design}
\label{universalDesign}
\textit{Universal design means designing, or accommodating, the main solution with regards to physical conditions, so that the solution may be used by as many people as possible, regardless of disability} \cite{Difi}. Designing the web portal according to the universal design principles was one of the most specific requirements from the customer. As the intended user group of SkalVi.no are children with special needs, the group aimed to design the web portal following these principles. 


\subsection{Scalability}
At the beginning of the project the group consulted with the customer about scalability for the web portal. The web portal was, for the period of this project, intended for the citizens of Trondheim municipality. However, it is desired for the project to become a national offer over time, thus the ability to scale was requested as a long term non-functional requirement.


\section{Changes to Requirements}
During the development of the product there have been some changes to the requirements. These changes were all decided with the customer during the customer meetings. The most substantial change was how the product would allow communication between users on the web page, and how to give the users the experience of a web society. Instead of creating a forum where the users could communicate, as intentionally planned, it was decided to implement the possibility to comment each activity. This was a decision based on the concern of users bullying one another, and how this eventually should have been managed. Furthermore, since the product was to be developed as a proof of concept this issue should not be excessively time consuming, and providing the possibility to comment was considered to be sufficient. Other changes to the requirements included providing the users a list of the providers from Aktørbasen (see section \ref{Aktordatabasen}). Thus, implement the possibility to filter and follow providers.

\subsubsection{Additions}
\label{additions}
Due to the changes to the requirements, there are some of the user stories which are added to the initial lists of stories. These are CH.04, CH.05, CH06, CH.06, PA.08 and PA.09 (see tables in section \ref{User stories}).

\subsubsection{Removals}
\label{removals}
A few of the original user stories were removed during the development (see table \ref{t:user_stories_removed}).

User stories US.02 and CH.02 were removed as a consequence of the user stories added, as these replace each other.

PA.04 has been removed due to feedback from users during the focus group (see section \ref{focusGroup}). This was initially a requirement the users desired, but as there are rarely any assistants at the activities this was considered irrelevant.

PA.05 and PA.07 are user stories which are proposed in the further development of the product, but not included in the group's version (see chapter \ref{further_development}). This was a decision the group made because of the prioritization from the customer, as these two stories had a relatively low priority compared to most other stories. As the users had expressed the importance of the children's experience of the concept, and especially their desire for a portal where children can find activities they can attend without help, the user stories concerning this user group were the topmost priority throughout the development. 

\begin{longtable}{@{\extracolsep{\fill}}
                |L{0.15\linewidth}
                |L{0.60\linewidth}
                |L{0.15\linewidth}|@{}}
\hline
\rowcolor{Gray}
\textbf{ID} & \textbf{User Story} & \textbf{Priority} \\
\hline
US.02 & \textbf{As a} user, \textbf{I want} experience a web-society feeling, \textbf{o that I can} invite friends, see participating friends, share, etc. The user is able to share activities on Facebook and invite Facebook friends to activities. & 80 \\
\hline
CH.02 & \textbf{As a} child, \textbf{I want} a place to communicate with others, \textbf{so that I can} plan activities, that aren't available,  Needs to be tested with users. & 60 \\
\hline
PA.04 & \textbf{As a} parent, \textbf{I want} to explicit know how many “assistants” there are on an activity, \textbf{so that I} know if my child will be taken care of, helped and included. & 100\\
\hline
PA.05 & \textbf{As a} parent, \textbf{I want} know how many participants / groups there are, \textbf{so that I can} know if they divide many participants into smaller groups, etc. &70 \\ 
\hline
PA.07 & \textbf{As a} parent, \textbf{I want} have access to more in-depth information about an activity than my child, \textbf{so that I can} figure out if the activity is suitable for my child. & 80 \\  
\hline
\hline

\caption{User Stories - Removed}
\label{t:user_stories_removed}
\end{longtable}


\cleardoublepage