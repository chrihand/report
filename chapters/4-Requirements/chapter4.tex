%===================================== CHAP 4 =================================

\chapter{Requirements}

\section{Minimum Viable Product}
\label{MVP}
The group represents their commitment and shared goal by specifying a definition of the Minimum Viable Product (denoted as MVP) that they are to deliver when the project deadline is due.

The MVP the group is to deliver is a web portal with a homepage, an overall activity page, admin and user profile pages. The Web portals home page presents upcoming events to the user. The portal is a user account service - meaning the events the homepage presents depends on the user logged in, such as 'my events'  if the user are hosting some. The portal will have an page that displays all the registered events, and allows for search and filtering. 
The design of the portal, will be under the WCAG 2.0 protocol, which state rules for universal design.

\section{Functional requirements}
This section contains the functional requirements for the product, which specifies what the system should do \cite{requirements}. These requirements were made by the group and verified by the customer, who also added some which he thought were important.

\subsection{User Stories}
\label{User stories}
The group were asked to create user stories (see table \ref{User_Stories}). The customer verified the user stories and added a priority to guide the group. To have a common understanding of how to define a story as ready or done the group created some criteria. 

\begin{description}
    \item[When a user story are ready:]
\end{description}
\begin{itemize}[noitemsep]
    \item A user story should be small
    \item A user story should be clearly defined
    \item A user story should be formulated according to As a \textless role\textgreater, I want \textless goal\textgreater So that \textless goal is achieved \textgreater
    \item A user story should be approved by the customer, product owner.
\end{itemize}

\begin{description}
    \item[When a user story is done:]
\end{description}
\begin{itemize}[noitemsep]
    \item A user story is done when all code is checked and approved by two others in the group
    \item A user story is done when all code is merged into the develop branch on GitHub
    \item A user story is done when all clarifications should be documented in wiki on GitHub
the design is approved by the responsible person on design, Ingrid Skar
    \item A user story is done when all code is approved in unit tests
    \item A user story is done when no more todos in Waffle.io on the case
    \item A user story is done when Functional testing is done
\end{itemize}

\begin{longtable}{@{\extracolsep{\fill}}
                |L{0.10\linewidth}
                |L{0.33\linewidth}
                |L{0.33\linewidth}
                |L{0.10\linewidth}|@{}}
\hline
\rowcolor{Gray}
\textbf{ID} & \textbf{User Story} & \textbf{Acceptance criteria} & \textbf{Priority} \\
\hline
US.00 & \textbf{As a} user, \textbf{I want} to see upcoming activities \textbf{So that I can} get an overview of my opportunities & The front page includes upcoming activities which the user can attend. The front page shows activities that a user can attend.& 100\\
\hline
US.01 & \textbf{As a} user, \textbf{I want} to see promo(video / pictures) of the activities \textbf{So that I can} see the purpose of the activity and what equipment is necessary, etc & & 100 \\
\hline
US.02 & \textbf{As a} user, \textbf{I want} experience a web-society feeling \textbf{So that I can} invite friends, see participating friends, share, etc. & Needs to be tested with users. & 80 \\
\hline
US.03 & \textbf{As a} user, \textbf{I want} to be able to log in \textbf{So that I can} sign up for events and see my personal page & A user must be able to log in. Get feedback after logged in. Stay logged in after refreshing page. Be able to log out. & \\
\hline
US.04 & \textbf{As a} user, \textbf{I want} a personal page \textbf{So that I can} see my upcoming activities & Be able to see personal information about myself. See the activities I have signed up for. & \\
\hline
US.05 & \textbf{As a} user, \textbf{I want} to be able to easily navigate between subpages on the webpage \textbf{So that I can} browse the webpage efficient & Navigation bar should be fully functional. Refreshing the webpage should not change the webpage. & \\
\hline
US.06 & \textbf{As a} user, \textbf{I want} to see as many activities as possible, not limited to one actor such as the municipality \textbf{So that I can} get an overview of my opportunities & & 100 \\
\hline
US.07 & \textbf{As a} user, \textbf{I want} to get a personalized view of activities \textbf{So that I can} get an overview of opportunities that fits me & Needs to be tested with users,. & 80 \\  
\hline

CH.00 & \textbf{As a} child, \textbf{I want} see relevant activities for me \textbf{So that I can} get an overview of my opportunities & & 70 \\
\hline
CH.01 & \textbf{As a} child, \textbf{I want} an overview of activities I am attending \textbf{So that I can} schedule my life & & 90\\
\hline
CH.02 & \textbf{As a} child, \textbf{I want} a place to communicate with others \textbf{So that I can} plan activities, that aren't available & Needs to be tested with users & 60 \\
\hline
CH.03 & \textbf{As a} child, \textbf{I want} to see all activities \textbf{So that I can} see what activities are available & The user should be able to view all activities & 100 \\  
\hline


PA.00 & \textbf{As a} parent, \textbf{I want} to see relevant activities for my child \textbf{So that I} see if it’s suitable for my child & & 100 \\
\hline
PA.01 & \textbf{As a} parent, \textbf{I want} to see all activities for my region in one place \textbf{So that I} don’t have to search the entire internet & & 100 \\
\hline
PA.02 & \textbf{As a} parent, \textbf{I want} to add activities I know about \textbf{So that I can} help others and to improve the selection of activities & & 90\\
\hline
PA.03 & \textbf{As a} parent, \textbf{I want} to give constructive feedback on what did/not work \textbf{So that} other parents can filter out activities & & 70\\
\hline
PA.04 & \textbf{As a} parent, \textbf{I want} to explicit know how many “assistants” there are \textbf{So that I} know if my child will be taken care of, helped and included. & & 100\\
\hline
PA.05 & \textbf{As a} parent, \textbf{I want} know how many participants / groups there are \textbf{So that I can}  know if there are many participants, but their divided into smaller groups, etc. & & 70\\
\hline
PA.06 & \textbf{As a} parent, \textbf{I want} have a informal contact channel \textbf{So that I can} quick and easy ask questions directly to the provider & Depends on whether the activity has a hotline that works & 100 \\
\hline
PA.07 & \textbf{As a} parent, \textbf{I want} have access to more in-depth information about an activity than my child \textbf{So that I can} figure out if the activity is suitable for my child & Navigation back and forward between the pages works. More information should be given at the “in depth” page. & 80 \\  
\hline

PR.00 & \textbf{As a} provider, \textbf{I want} to easily register detailed information regarding facilitation for each activities. \textbf{So that} others can find my activities & & \\
\hline
PR.01 & \textbf{As a} provider, \textbf{I want} to see and show ratings on my activities \textbf{So that I can} improve my activities & & 80 \\
\hline
PR.02 & \textbf{As a} provider, \textbf{I want} to easily register my events in one and only one place \textbf{So that} maintenance and duplication is avoided & & 100\\
\hline
PR.03 & \textbf{As a} provider, \textbf{I want} the opportunity to add picture / video for an activity
 \textbf{So that I can} present the activity in a good way & Can also be for users, e.g. parents who want to add photos & 70 \\
\hline

SM.00 & \textbf{As a} system maintainer, \textbf{I want}  easily install the system in a standard web hotel \textbf{So that}  & & 100 \\
\hline
SM.01 & \textbf{As a} system maintainer, \textbf{I want} to be able to make the system easily available through a standard domain name \textbf{So that I can} improve my reach & & 100 \\
\hline
SM.02 & \textbf{As a} system maintainer, \textbf{I want} to be able to secure my data \textbf{So that}  users can trust me & E.g. having DB on a secure server & 100\\
\hline
SM.03  & \textbf{As a} system maintainer, \textbf{I want} be able to integrate with new event systems systems, e.g. Facebook, Events Trondheim... \textbf{So that}  users can add a variety of events easily & & 100 \\
\hline

\caption{User Stories}
\label{User_Stories}
\end{longtable}



\subsection{User Cases}
\label{User cases}
The use cases\cite{} created by the group are based on the requirements of the system. The use cases are represented alongside their respective figure, through figures x.xx - y.yy. Prerequisite in all use cases is a device with Internet connection and a  Internet browser.

\begin{figure}
    
\end{figure}
 
\section{Non-functional requirements}
Non-functional requirements specifies how the system performs a certain function \cite{requirements}. The non-functional requirements were defined by the customer.


\subsection{Open Source}
Open source means that the source code to the product is made available to everyone, usually on the Internet. Open source code can be published under many different licence agreements.

\subsubsection{Apache2 Licence agreement}
The customer required that the product was published as open source under the licence Apache2 \cite{apache2}. Apache2 is a licence that does not restrict the re-usage of the source code. The source code is available to all, and can be reused, further developed, distributed, and more as long as the required original notices follows the source code and is maintained. 

The notices has to be in a text file called NOTICE.txt, that the work includes as part of its distribution. All that means is that the only thing needed to re-use and deploy another version of the www.skalvi.no source code, is to include the original notice (the www.skalvi.no.notice) in the re-distribution.


\subsection{Server Availability}
Server availability became a non-functional requirement in the second sprint, when the customer requested that the latest stable development version was available on www.skalvi.no at all times. The customer also wanted the server to be updated at least once a week. The group decided that the server would be updated two times a week, Wednesday and Fridays. Because the development methodology was scrum, this requirement was made aware of early in the  development process, and could be incorporated in the working sprint. This reveled that a few changes to the architecture was needed. and potential big problems, such as production- v.s. development configurations, class dependencies, etc  was discovered early and was fixed and avoided.

\subsection{Privacy}
As the web portal was centered around users and the information they supplied, requirements for privacy became important.

\subsection{Universal Design}
\label{universalDesign}
"Universal design means designing, or accommodating, the main solution with regards to physical conditions, so that the solution may be used by as many people as possible, regardless of disability." \cite{Difi} Designing the web-portal according to the universal design principles was one of the most specific requirements from the customer. As the intentional user group of www.skalvi.no is children with special needs, the group aimed to design the web-portal following these principles. Examples on principles that were implemented are;   


\subsection{Scalability}
At the start of the project the group asked the customer about scalability for the web portal. The web portal is initially intended the citizens of Trondheim municipality, so the customer requested that this should be a long term non-functional requirement.


\section{Changes to Requirements}
Remember:
- Forum
+ Comments

\subsubsection{Additions}

\subsubsection{Removals}


\cleardoublepage