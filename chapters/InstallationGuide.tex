%===================================== CHAP 6 =================================

\lstset{ 
    language=bash, % choose the language of the code
    basicstyle=\fontfamily{pcr}\selectfont\footnotesize\color{black},
    keywordstyle=\color{black}\bfseries, % style for keywords
    numbers=none, % where to put the line-numbers
    numberstyle=\tiny, % the size of the fonts that are used for the line-numbers     
    backgroundcolor=\color{lightgray},
    showspaces=false, % show spaces adding particular underscores
    showstringspaces=false, % underline spaces within strings
    showtabs=false, % show tabs within strings adding particular underscores
    frame=false, % adds a frame around the code
    tabsize=2, % sets default tabsize to 2 spaces
    rulesepcolor=\color{gray},
    rulecolor=\color{black},
    captionpos=b, % sets the caption-position to bottom
    breaklines=true, % sets automatic line breaking
    breakatwhitespace=false, 
}


\chapter{Installation guide}

\section{Preface}
In this guide, the operating system of the server is Ubuntu 16.04, with Nginx as reverse proxy. The steps are the same on all operating systems, but the terminal commands may be different.
All commands are typed in the terminal / command line, unless explicitly stated. Some of the commands may need to be executed with "sudo"

\subsection{Prerequisites needed}
\begin{itemize}  
\item Python 3.5.x
\item Node.js version 6.10.x
\item RAM 1Gb (minimum)
\item Git
\end{itemize}

\subsection{Recommended tools}
\begin{itemize}  
    \item virtualenv, for Python3
    \item virtualenvwrapper for Python3
    \item Nginx as reverse proxy server
\end{itemize}

\section{Installation Guide}
\begin{enumerate}
    \item Clone the UngIT github repository to your server.    
    \begin{lstlisting}
        $ git clone https://github.com/ung-it/UngIT.git
        $ cd UngIT # change into the project directory
    \end{lstlisting}
    
    \item Install application dependencies. To be able to execute these commands, you need to be in the project directory root. To check this, find the directory level containg the files manage.py, package.json and py-requirements.py. In the terminal run the following commands
    \begin{lstlisting}
        $ npm install  # (may need sudo as prefix)
        $ pip install -r py-requirements.py
    \end{lstlisting}
    
    \item Now that you have installed the needed project dependencies, the next thing is to create the bundle file. The bundle file is used to minify and distribute the JavaScript files.    
    \begin{lstlisting}
        $ npm run build-production # (may need sudo as prefix)
    \end{lstlisting}

    \item After creating the bundle file, we have to configure the database.
    \begin{lstlisting}
        $ Python manage.py migrate
    \end{lstlisting}
    
    if this prints something in read, and asks you to execute the command 
    "\$ Python manage.py makemigrations --merge"
    you have to execute it, and then run the first command again.   
    \item Next up, is to collect all the static files the application needs, in one directory. that the application looks for all the files while running.
    \begin{lstlisting}
        $ Python manage.py collectstatic
        $ yes  (to overwrite files)
    \end{lstlisting}
    
    \item All that remains now is to start the application and make it run, even if the terminal is closed. We will make this happen by making the application run as background process
    \begin{lstlisting}
        $ Python manage.py runserver &
        $ ctrl + c 
    \end{lstlisting}
    
    \item To stop the application, we need to find the process id, then signal a kill for that application.
    \begin{lstlisting}
        $ Top | grep python3
        >> <pid>  xxx xxx xxx xx
        
        $ kill -9 <pid>
    \end{lstlisting}
    

\end{enumerate}
    

\cleardoublepage