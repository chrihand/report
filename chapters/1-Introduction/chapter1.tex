%===================================== CHAP 1 =================================

\chapter{Introduction}

\section{Course}
The course, IT2901 Informatics Project II, is a group project where students work in groups to carry out a software project \cite{EmneKode}. The project goal is for students to use experiences from previous courses at NTNU to develop a product for a real customer. The development process of this product is documented in this paper.


\section{Project Description}
UngIT is a project that addresses the problem \textit{“Children with special needs often have difficulties finding and getting involved in leisure activities outside school”} \cite{SintefBachelorProjectDescription}. The task given is to develop an activity portal for children with special needs between the age 10 to 15 years old. The project customer is SINTEF Digital and Trondheim Municipality, and the customer wants a web-based portal for allowing children with special needs to search for, find and join leisure activities.

This project will chart the dissemination of existing leisure and identify commercially available welfare technological solutions so that children with disabilities can easier participate in, and master leisure activities. The activities are intended to support the individual habilitation and rehabilitation processes \cite{SintefOnlineProjectDescription}.

The project is to be developed as a “Proof of Concept"; and will give difficulties regarding universal design, representing the correct data without excluding any parts of the target audience, creating a social network and gathering all the data that is available in one portal. Thus the project requires the group to do a lot of research and user testing. Also the project requires the group to talk to the providers and different user groups of leisure activities to gather requirements. The impact of this project is that the activity providers get one common channel for advertising leisure activities. Children with special needs and their parents get one instead of many channels to find activities. Thus the primary function of the project is to create a common portal for leisure activities.

Application requirements is not given explicitly by customer, and is to be defined through workshops with users and providers. Note that developing a common database for all the leisure activity providers and their activities is not part of the project, but the web portal needs to handle the possibility of a connection to this sort of database.  

\section{Customer}
As mentioned in the project description, the customer for this project is Trondheim Municipality and SINTEF Digital.

Trondheim Municipality \cite{TrondheimMunicipality}, is the product owner. Their focus is on the citizens and therefore, they have established the welfare technology program \cite{WelfareProgram}. Their vision is “Safe where you are!” and their wish is to develop and use welfare technology so their citizens can feel safe and experience success wherever they are.

SINTEF Digital \cite{SintefDigital}, is a research division at SINTEF. SINTEF Digital focuses on user-centered development and carries out research in Information and Communication Technology.

All communication about the project and product is performed either over mail or through meetings with Babak Farshichan, he works at SINTEF Digital.


\section{Group Members}
The group consists of six members; Christina Hellenes Andresen, Andreas Norstein, Sigve André Evensen Skaugvoll, Ingrid Skar, Martin Stigen and Thomas Wold. The members started their Informatics Bachelor degree in August 2014 and has through the courses at Norwegian University of Science and Technology (denotes as NTNU) gained skills in both programming and software development methodologies.

\textbf{Christina Hellenes Andresen}\\
Andresen has prior experience in programming from courses taught at NTNU, with most experience in web development. She has some knowledge with writing papers, development methodologies and experience with testing.

\textbf{Andreas Norstein} \\
Norstein has gained experience in programming from courses at NTNU and Queensland University of Technology (denoted as QUT), with experience in Java, Python, C\#/.NET, Nodejs and Javascript. He has experience from server setup and back end coding from courses at QUT, with focus on deployment to cloud services. 

\textbf{Sigve André Evensen Skaugvoll} \\
Skaugvoll has prior experience in Nodejs, Angular 2, Java and Python from courses during the Bachelor of Informatics studies at NTNU. He has additional experience in server and back end development, such as Nginx and reverse proxy servers, as well as testing, gained from other projects. He has also experience from environments such as Linux, Unix, and Windows.

\textbf{Ingrid Skar} \\
Skar has acquired experience in programming from courses taught at NTNU and QUT, primarily in web development. She has also experience from server setup and back end coding, focusing on deployment to cloud services.

\textbf{Martin Stigen} \\
Stigen started learning the programming language "ActionScript 3" in high school. From courses taught at NTNU he has gained experience in Java, Python, databases and web development. He has spent much of his leisure time further developing these skills. Stigen has also worked as a student assistant multiple times in the programming courses TDT4100 and TDT4110.

\textbf{Thomas Wold} \\
Wold has gained experience in programming languages such as Java, Python and Javascript from courses taught at NTNU. Additionally he has gained experience in web development, databases, software security and software development. In his spare time he has learned the programming language Fuse and has knowledge about mobile app development.

\cleardoublepage