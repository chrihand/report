%===================================== CHAP 3 =================================

\chapter{Project Management}

\section{Methodology}
\label{methodology}
During the development of the product the group decided to utilise scrum and the agile methodology. This was a natural choice of methodology as the customer required the process to be iterative and based on the group's previous experiences.

The process methodology was adjusted to suit the project in the best possible manner, as well as to suit the group's needs. The group completed sprint reviews, sprint retrospectives and daily meetings to achieve the best possible outcome of the project. The first 15 minutes of the daily meetings was used to set an agenda and goal of the day, and was completed by using the last 15 minutes to summarize the day. The product owner of the project is also the project's customer, and will only be referred to as the customer throughout this document. To maintain consistent communication with the customer, there were scheduled stand-up meetings bi-weekly or if needed. These meetings allowed the group to continuously update the customer and demonstrate prototypes of the product. This ensured that the product developed corresponded to customer's vision. To aim for a product covering the end users needs, the customer provided a set of test users, to test the product iteratively during the development.

During the development process the group also utilised the principles of Extreme Programming, specifically the practice of pair programming. A decision based upon the groups desire to maximize the learning opportunities from each other resulting in the best possible outcome. 

\subsection{Development}
Throughout the development the group utilised the practice of issue tracking on GitHub (see chapter  \ref{GitHub}), and structured the tasks based on a roadmap using milestones. To ensure that the group followed the principles of scrum - it was decided to define the milestones as a correspondence to time, with each sprint as a milestone in the project.

To easily track the issues during the development the group used waffle.io (see chapter  \ref{Waffle.io}) as a scrum board. This provided the group a visualisation of the process - with issues in 'product backlog', 'sprint backlog' and 'in process' including who is working on which, using assignees. 

\subsection{Report}
Waffle.io was also used during the process of writing the report - primarily to ensure that every member participated in writing the report, and every member needed to proof-read each part before it were categorized as done.  


\section{Team Organization}\label{projectOrganisation}
The group decided to divide the project into reasonable areas - and delegate the main responsibility to two of the group members, where one had the primary responsibility. This would ensure that all parts of the project were kept in mind during the development, as well as more than one person knew what was going on in one specific area.

\subsubsection{Scrum Master}
Responsible: Ingrid Skar \\
Skar were given the role as Scrum Master due to previous experience with leading a team and the coordination of scrum related tasks. This includes coordinating workpackages, sprints and milestones.

\subsubsection{Group Leader}
Responsible: Christina Hellenes Andresen \\
Andresen were voted to have the role as group leader, this includes attending group leader meetings. The group also decided that the group leader were to create meeting agendas and lead the group meetings. 

\subsubsection{Contact Responsible}
Responsible: Sigve André Evensen Skaugvoll \\
The responsible person for keeping contact with both customer and supervisor were Skaugvoll. He were responsible of arranging meetings and sending them the meeting agenda before meetings.

\subsubsection{Architect}
Responsible: Martin Stigen \\
The role of architect were given to Stigen, due to interests and experience. The architect were responsible for providing initial draft to the architecture and for integrating the back end together with the front end. Skaugvoll were also a part of the architect team.

\subsubsection{Database}
Responsible: Andreas Norstein\\
Norstein were given the main responsibility for the database, this responsibility contains investigating the different options and needs. Norstein had the responsibility together with Skaugvoll.

\subsubsection{Back End}
Responsible: Sigve André Evensen Skaugvoll\\
The back end responsible were responsible for investigation the back end options. Norstein worked together wih Skaugvoll after the group had decided on using Django to set up the development environment.

\subsubsection{Front End}
Responsible: Thomas Wold\\
Wold were given the role of front end responsible, this were mainly because of experiences with front end development from courses he had attended at NTNU. This responsibility were shared with Skar and Stigen. Their work included choosing development language, framework and libraries, as well as preparing the development environment.

\subsubsection{Graphical User Interface Designer}
Responsible: Ingrid Skar\\
The graphical user interface designer were responsible for creating the paper prototype and styling the product. Skar were given the main responsibility to this task, together with Andresen, mainly because of interests in this field.

\subsubsection{Testing}
Responsible: Sigve André Evensen Skaugvoll\\
Skaugvoll were given the role as tester, this includes to set up test strategy and plan. Andresen and Skar were co-responsible for testing. 

\subsubsection{Report}
Responsible: Christina Hellenes Andresen\\
Andresen were given the role of report responsible, due to previous experience. This role includes setting up the table of content and keeping the report backlog on Waffle.io, \ref{Waffle.io}, updated. Wold were also a part of this team.


\section{Work Breakdown Structure}


\subsection{Gantt Diagram} 

\section{Risk Analysis} \label{riskAnalysis}
The risk analysis was originally created during sprint iteration 0. The cases and their values were mainly based on previous experience from other projects. Then they were adapted to this project by thoroughly going through each case collaboratively, doing new evaluations.

Cases are ranked from 1: Minimal likelihood but may occur, to 9: Very likely. They are also sorted on importance descending.
\begin{longtable}{@{\extracolsep{\fill}}|L{0.14\linewidth}
                |L{0.09\linewidth}
                |L{0.09\linewidth}
                |L{0.14\linewidth}
                |L{0.15\linewidth}
                |L{0.15\linewidth}|@{}}
\hline


\rowcolor{Gray}
\textbf{Description} & \textbf{Likelihood (1-9)} & \textbf{ Impact (1-9)} & \textbf{Importance {\footnotesize (Likelihood * Impact)}} & \textbf{Preventive Action}    & \textbf{Remedial Action} \\ \hline


Conflicts in group & 7 & 7 & 49 & Talk together, give constructive feedback & Try to resolve it with a neutral third party \\
\hline
Unrealistic goals & 7 & 6 & 42 & Talk to customer, be realistic & Be flexible and change goals \\
\hline
Falling Behind Schedule & 6 & 6 & 36 & Daily stand-up meetings, be efficient at meetings, have scheduled meetings & Re-estimate workload, increase work hours \\
\hline
Sickness & 8 & 4 & 32 & Eat healthy, be hygienic & Stay at home, if possible work from home \\
\hline
Technical challenges & 5 & 6 & 30 & Think ahead, test early, build modular & Reconsider technical choices, contact people with knowledge \\
\hline
Lack of knowledge & 9 & 3 & 27 & Be prepared, read up on topic/tech & Ask for help, read about the topics we miss knowledge about \\
\hline
Merge conflicts & 8 & 3 & 24 & Always pull from master before branching & Solve it, run unit tests after resolving conflict \\
\hline
Absent customer & 3 & 7 & 21 & Have scheduled meetings & Refer to customer contract \\
\hline
Lack of communication in group & 5 & 4 & 20 & Slack and meetings & Talk to group representative \\
\hline
Lack of responsibility & 3 & 6 & 18 & Daily stand-up meetings, follow up on given tasks, collaborative work hours & Group leader’s responsibility to follow up \\
\hline
Inefficient meetings & 8 & 2 & 16 & Follow the agenda for the meeting, strict group leader, bring coffee & Take breaks \\
\hline
Late Arrivals & 8 & 2 & 16 & Give reminders, use the calendar & Cake punishment, contact supervisor if it repeats \\
\hline
Data loss & 2 & 6 & 12 & Save often, use git, commit and push often, have local copies & Increase workload, redo work \\
\hline
Poor execution of methodology & 4 & 3 & 12 & Strict group leader, scrum master, quick recap of methodology & Recap of methodology \\
\hline
Absence & 2 & 5 & 10 & Give reminders, use the calendar & Cake punishment, contact supervisor if it repeats \\
\hline
Disagreement of priorities & 5 & 2 & 10 & Everyone gets the opportunity to say what they mean & Talk to customer \\
\hline
Loss of group members & 1 & 9 & 9 & Group representative, social activities & Distribute workload equally over the entire team, adjust goals of project \\
\hline

\caption{Risk analysis}
\label{risk_analysis}
\end{longtable}



\section{Sprints}
\label{sprintsAndMilestones}
The group divided the development process into sprints, were each sprint lasted for two weeks running Monday to Friday, and planned with respect to the given milestones. 

\begin{longtable}{@{\extracolsep{\fill}}
                |L{0.16\linewidth}
                |L{0.18\linewidth}
                |L{0.12\linewidth}
                |L{0.40\linewidth}|@{}}
\hline
\rowcolor{Gray}
\textbf{Dates}&\textbf{Planned Work Hours}&\textbf{Sprint}&\textbf{Goal}\\
\hline
06.02 - 10.02&136.5&Sprint 0&Initialization phase. Planning, product backlog set up and first draft design.\\
\hline
13.02 - 24.02&270&Sprint 1& Deliver first interactive prototype to customer.\\
\hline
27.02 - 10.03&270&Sprint 2&-\\
\hline
13.03 - 24.03&270&Sprint 3&-\\
\hline
27.03 - 07.04&270&Sprint 4&-\\
\hline
18.04 - 28.04&270&Sprint 5&Find and fix bugs.\\
\hline
01.05 - 30.05&360&Sprint 6&Final touch report, first aid kit and deliver\\
\hline
\caption{Sprints}
\end{longtable}


\subsection{Workdays}
The group had workdays four times a week, see table \ref{Time table}. The group decided together on all hours it was expected to work on the project jointly. The group also decided to sit together and work at these hours, these hours were sat at times were no one had lectures or other activities. The group wished to sit together, because it was easier to have an overview of what the rest of the group were working on and it was easier to get help from each other when the group sat together. During the workdays the group members worked on individual tasks.

\subsection{Daily Meetings}
The group held daily meetings on Mondays and everyday before a workday. 


\begin{longtable}{|l|l|l|l|l|l|}
\hline
\rowcolor{Gray}
&\textbf{Monday} & \textbf{Tuesday} & \textbf{Wednesday} & \textbf{Thursday} & \textbf{Friday} \\
\hline
08:15 - 09:00 &               &               & Meeting &               &               \\
\hline
09:15 - 10:00 &               &               & Workday       &               &               \\
\hline
10:15 - 11:00 &               & Meeting & Workday       & Meeting & Meeting \\
\hline
11:15 - 12:00 &               & Workday       & Workday       & Workday       & Workday       \\
\hline
12:15 - 13:00 &               & Workday       & Workday       & Workday       & Workday       \\
\hline
13:15 - 14:00 &               & Workday       & Workday       & Workday       & Workday       \\
\hline
14:15 - 15:00 & Meeting &               & Workday       &               & Workday       \\
\hline
15:15 - 16:00 & Meeting &               & Workday       &               & Workday\\
\hline
\caption{Weekly Time Table}
\label{Time table}
\end{longtable}


\subsection{Customer Meetings}
\subsection{Supervisor Meetings}

\section{Milestones}
The milestones are the important dates given in the project and an internal deadline. In addition to this, the milestones consist of the project's phases - in accordance to GitHub's guidelines \cite{GitHubGuide}. 

\begin{longtable}{|l|l|}
\hline
\rowcolor{Gray}
\textbf{Dates} & \textbf{Milestone} \\
\hline
15.02 & Very preliminary delivery of report \\
\hline
19.02 & Mid semester version of report\\
\hline
07.04 & Group deadline. No new features should be implemented. \\
\hline
08.05 - 12.05 & Final Presentation \\
\hline
30.05 & Deliver  \\
\hline
\caption{Milestones}
\end{longtable}

\subsection{Sprint Backlog}




\cleardoublepage