%===================================== CHAP 2 =================================

\chapter{Prestudies - Process}

\section{Development Process} \label{s:development_process}
The group chose to utilize Scrum \cite{scrum} as the development framework during the process of developing the product. \textit{"Scrum is an iterative and incremental agile software development framework for managing product development"} \cite{scrumWikipedia}. Because of frequent meetings with the customer and changes to customer requirements, Scrum was a suited choice of framework for the project. All group members had previous experience using the Scrum methodology from other projects, which gave the group an advantage of an quick start. 

Other frameworks that were taken to consideration were Extreme Programming and Kanban. Both of these frameworks are commonly used in software development, and are both classified as agile processes. The processes of these two frameworks are less definite than Scrum, and does not provide all the features Scrum does. The group concluded that Scrum, including parts other frameworks to adjust it to the project, was the best choice (see section \ref{methodology}). Furthermore the customer wanted an agile development process, and required documentation if someone else were to continue the project after the group's completion. 

\section{Tools}
\label{tools}
The following section will contain information about all tools utilized in this project, including tools for project management, communication, document sharing and product development.

\subsection{Project Management and Communication}

\subsubsection{Slack}
The group chose Slack \cite{Slack} as the main communication channel within the group. Slack allows members to create distinct threads, where discussions about different subjects may be raised. This makes communication easier and organized, compared to other chat-applications like Facebook Messenger. Slack also provides great integration with GitHub, which kept all group members updated on changes in the GitHub repository. 

\subsubsection{Email}
Email was the selected communication channel between the group, the group supervisor and the customer.
Email was chosen because it is easy to maintain, as the sender and receiver have their own copies. Furthermore, the possibility to send attachments, categorize on subject and unlimited amount of characters was a motivation. Both the customer and supervisor preferred email as their main channel to communicate with the group.  

\subsubsection{Toggl}
Toggl \cite{Toggl} was selected as the group's time tracking tool. It makes it easy to track time, and specifies what each group member have been working on. Toggl provides great visualization of time spent on each task, with different types of charts and diagrams. The statistics are open for the entire group, which means every group member can see what other members are working on. Toggl offers the use of tags, making it convenient to divide group tasks. This allows for tracking the progress, and to see which types of tasks are most time consuming.  

\subsubsection{Git}
The group chose to use Git \cite{Git} as the version control system. Git is an open source system, that allows teams to work on projects efficiently and safe. Git makes it easy to manage project source code history. It replaces Source Code Management (denoted as SCM) tools because of cheap local branching, convenient staging areas, and multiple workflows. Git was chosen as version control instead of SCM tools, because it allowed the group to use a workflow called git-flow \cite{gitFlow}. Git-flow suits the project and the development methodology with release-, development- and feature branches. 

\subsubsection{GitHub}
\label{GitHub}
GitHub\cite{GitHub} is a hosting service for distributed revision- and version control systems. GitHub offers SCM functionality and access control. Furthermore, it provides collaboration and project features; such as wiki-pages, task management tools, and both private and public repositories. The groups decision to use GitHub was based on customer's demand, due to the project was required to be open source and hosted on GitHub. 


\subsubsection{Waffle.io}
\label{Waffle.io}
The group chose Waffle.io \cite{Waffle} as the main task manager tool. The product backlog and all sprint backlogs, were managed through Waffle.io. Additionally it has has great integration with GitHub, and automatically synchronizes tasks with issues on GitHub. This was the main reason for why the group chose to work with Waffle.io compared to other known tools like Trello. The interface is clean and easy to use. It displays tags and subjects in a clear way, and who is working on which issue. It also supports commenting on issues, making it easy to keep discussions on issues relevant. 

\subsubsection{Gantt Project} \label{sss:Gant_Project} 
Gantt Project \cite{Gantt} is a free desktop scheduling and management app, which provided the group a convenient approach to create a Gantt Diagram (see section \ref{GanttDiagram}). 

The group selected to utilize Gantt Project to create the Gantt Diagram because of the many features provided by the app; such as efficiently adding tasks and sub-tasks, and set start- and finish date for each specific task. The app supplied the group with a visualization of the project life cycle based on tasks and dates added.  


\subsubsection{Google Drive}
For document sharing, the group decided to use Google Drive \cite{GoogleDrive}. Google Drive is great for sharing documents and information within a group. It also allows the group members to simultaneously write and edit the same documents in real time, making it easy for all members to collaborate. Google Drive was primarily chosen because of their convenient web solution, which other competitors like Dropbox or OneDrive did not support. Other features that Google Drive offers is the integration with other applications and document types, such as .docx, spreadsheets and powerpoints. The support for multiple file types has been essential in the product development.

\subsubsection{ShareLaTeX and LaTeX}
ShareLaTeX \cite{ShareLatex} is an online editor for writing text documents in the typesetting system LaTeX, created for writing scientific and technical documents.   

ShareLaTeX allows collaborative work in real time, and provides a convenient way of dividing the document into smaller text files. Subsequently, documents are easy to construct and maintain, even when the document size expands. Compared to other online editors, such as Google Documents, ShareLaTeX provides better performance on large documents. The decision was also based on the possibility to store back-ups in GitHub and Dropbox.
  
\subsubsection{PyCharm}
PyCharm\cite{PyCharm} is an Integrated Development Environment (denoted as IDE) for professional Python developers. The group chose to use PyCharm because it supports production of code more efficiently, and provided good error checking. The entire group also used the professional edition, providing additional support for git, web development and the Django framework, resulting in an IDE supporting the entire project.

\subsection{Front End}
\label{frontEnd}
\subsubsection{React}
For front end development the group chose to use components defined in the ReactJS library \cite{React} combined with HTML-files created using Django templates (see section \ref{django}). The React library offers easy creation of components that are dynamic, fast rendering, and user friendly. This made it possible to create responsive and fast loading web pages. Using React, the overall structure of the project also became organized, easy to understand and update. It also allowed much reuse of code, saving time and complexity. 

\subsubsection{Redux}
\label{redux}
To ensure that the application behaved consistently the group decided to use Redux. \textit{"Redux is a predictable state container for JavaScript apps"} \cite{Redux}. By using Redux the group managed to cope with state mutations, asynchronicity, and manage the state of the data in the application. An alternative to Redux was Flux, but it does not handle asynchronicity and was therefore not a suitable choice. 

The group utilized Redux with the React components. It was primarily used to filter and search for the activities presented to the users, and assure consistent state among the application.  

\subsubsection{Bootstrap}
Another tool used for front end development was Bootstrap \cite{Bootstrap}. This is one of the most used HTML, CSS and Javascript frameworks to develop responsive and mobile websites. The group chose to use Bootstrap because it saves time by providing predefined CSS-rules and components. This also reduces the code complexity and size. 

\subsubsection{Material Design Lite}
\label{mdl}
Material Design Lite \cite{Material_Design_Lite} is a library for designing dynamic websites, and does not rely on any other Javascript frameworks. \textit{"It aims to optimize for cross-device use, gracefully degrade in older browsers, and offer an experience that is immediately accessible."} \cite{Material_Design_Lite}. The group chose to use Material Design Lite because it helps to create a complete and modern design, behaves consistent on different platforms, and provides a rich library of predefined components.

\subsection{Back End}
\label{backEnd}

\subsubsection{Django}
\label{django}
Django was chosen as the main framework for writing the back end of the web portal \cite{django}. Django is a high-level framework written in Python for developing websites. All group members had previous experience with Python, and combined with the Django framework, this allowed rapid development. Furthermore Django is a framework that manages most of the the back end specific programming; such as log in, database connection and routing. This allowed the group to concentrate on other features of the application. 

\subsubsection{Python}
\label{python}
The group chose to use Python as the programming language for the server side, due to the choice of Django as the main framework \cite{python}. Since Python is one of the most common programming languages \cite{Programminglanguagespopularity} and has a huge community, it met the requirement of making further development of the product easy.

\subsubsection{SQLite}
The chosen database for the web portal was SQLite \cite{SQLite}. It was chosen because it does not return generic SQL and raw table content. SQLite returns high-level and specific data that does only contain the essential information and suits the purpose of the web portal without excessive features. It is reported that SQLite is often faster than a client/server SQL database engine in this scenario. \cite{Server-sideDatabase} 

\subsubsection{Amazon Web Services - EC2}
Amazon Web Services (denoted as AWS) is a cloud services platform, offering computing power, database storage and more \cite{AWS}. The customer chose AWS as the Virtual Private Server (denoted as VPS) provider because Amazon is the major provider on the market. The customer requested that the production data was safe and would not disappear. AWS provides this security, and thus fulfill this requirement.

The group choose to use Amazon Elastic Compute Cloud (denoted as EC2) \cite{EC2}. It is a web service that provides complete control of the computing resources, and reduces time spent to create a VPS instance. The decision was also based on the fact that members of the group had prior experience with EC2 technology. 

\subsubsection{Nginx}
The group chose to use Nginx as a reverse proxy and load balance between the EC2 instance and the web portal application \cite{nginx}. The use of reverse proxy can hide the existence of an origin server, and reduce the stress-load on the web portal. The group decided to use Nginx instead of other reverse proxy, such as Apache2, because Nginx is the newest solution available, and the possibility for asynchronous-, non-blocking- and event-driven connections have been improved.

\subsubsection{uWSGI}
uWSGI is a deployment option on servers like Nginx, and aims at full stack development for building hosting services \cite{uWSGI}. The reason uWSGI was used in this project is because \textit{"A traditional web server does not understand or have any way to run Python applications."} \cite{whyUseWSGI}. Thus, the group decided to use uWSGI because it can communicate with Nginx and the python application. This provided the web portal with faster response time on user requests. The group decided to go with uWSGI instead of similar solutions such as Gunicorn, because uWSGI is easier to get started with, and allows for easier and more configuration.

\cleardoublepage