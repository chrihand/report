%===================================== CHAP 2 =================================

\chapter{Prestudies}

\section{Research}
In advance of starting the development of the project, a thorough research had to be completed.

Initially the group researched what type of methodology would suit the project, the customer and the group it self (see section \ref{methodology}). When the methodology the group would use during the development was chosen, the group researched what tools facilitated an effortless use of this. The group members had a lot experience in using different tools, which opened for many opportunities for deciding what should be used (see section \ref{tools}).

One of the requirements from the customer was a web portal applying the principles of universal design (see section \ref{universalDesign}). This was an aspect few of the group members had prior experience with. Taking this into account the customer provided a resource, DIFI - Universal design principles \cite{Difi}. During the development process, design choices, such as focus marking and colours, have been influenced on knowledge acquired utilising this resource. 

The customer provided several alternative solutions (see section \ref{alternativeSolutions}) for the group to explore. This was primarily done to see how to best satisfy the users' needs. To give the group the best possible understanding of the users' needs there was arranged a workshop together with the users and the providers (see section \ref{WorkshopUserAndProviders}). 


\subsection{Workshop With Users and Providers}
\label{WorkshopUserAndProviders}
During the researching phase; two of the group members, Andresen and Skaugvoll, attended a workshop to map the requirements the users had to the web portal. 

The workshop was divided into two parts; one with the providers of the activities which are to be posted on the site, and one with the children and parents which would use the site to find the respective activities. The workshop revealed the different needs the users had to the web portal; especially emphasizing how these differed based on user type, but all functionality should be accessible from the same portal. 

Based on the knowledge Andresen and Skaugvoll acquired throughout the workshop the group got a more definite idea of what should be developed. See appendix \ref{workshop_one} for a summary of the workshop.


\section{Customer input}
\label{Customer input}
During the first meeting with the customer, the group received input about the overall structure of the desired product, as well as the research required. The customer emphasized the fact that the project was a “Proof of Concept”, and that the group had to research utilizing universal design principles in web applications.

The application to be developed is purposed to be used in Trondheim municipality, with the possibility to be scaled up to a bigger user group. Trondheim municipality desires to easily allow parents and youths to find, participate and master leisure activities. This user base represented a big group of users, and the project was therefore allocated resources to perform workshops and user tests.

Both server development and graphical user interface is fundamental for the group to create the web portal the customer desires. The group had a relative free choice when it came to technology to be used during the development, but customer input had an impact on the final decisions made.

\section{Exsisting Solutions}
\label{alternativeSolutions}
The user group of skalvi.no has a limited amount of existing solutions to choose from when it comes to finding leisure activities offered. They are not satisfied with the existing solutions, because the activities offered is not easily accessible and there are often lack of detailed information about the activities.  
The contemporary solutions are TRDevents, KUBA, Beitostølen Helsesportsenter and GoSmart.

\subsection{TRDevents}
In Trondheim, TRDevents \cite{TRDevents} is the main portal to find activities and other events. Events and activities can be registered both by organizations and private people. Their focus is to list all events in Trondheim, and it is meant to keep everyone updated about upcoming events and activities. 

TRDevent lists all activities and event on one page in a list with approximately 50 items per page. If a user wants to see more, the user have to scroll through the whole list til be able to click on a button to replace the existing 50, with 50 new activities and events. 
This solution support both searching and filtering on date and category and it is possible to share on Facebook. For each activity and event, it is possible to export it to Google calendar and/or Apple calendar, and the user gets a representation of related events. 

\subsection{Kultur for Barn (KUBA)}
A portal which aims towards this project's user group is KUBA \cite{KUBA}. KUBA stands for "Kultur for Barn" (Culture for children), and is Trondheim municipality's cultural program for children in all ages. KUBA strives to be as diverse and as inclusive as possible, such that everyone can participate in their activities. KUBA is not a portal on its own, but is a program offered by Trondhim municipality and is therefore just a part of their website. KUBA displays their activities with a pdf-file that is linked on their web-page and they use Facebook. KUBA does not have any other functionality.

\subsection{Beitostølen Helsesportsenter}
A national offer is Beitostølen Helsesportsenter (further denoted as BHSS) \cite{BHSS}. BHSS provides people in all ages with physical, psychological and/or cognitive disabilities the opportunity to rehabilitation through activities. The rehabilitation takes place at their center in Beitostølen located in Øystre Slindre municipality. Their main focus is on opportunities rather than limitations.  
BHSS does not display their activities in any specific way, but they display the the different opportunities by categories. To be able to attend these activities, users have to apply for a rehabilitation-stay through a doctor. BHSS has therefore no functionality like sign-up on activities on their website.  

\subsection{Go Smart}
A offer outside of Norway that is similar to skalvi.no is Go Smart \cite{GoSart}. Go Smart is an initiative from The National Head Start Association (further denoted as NHSA) \cite{NHSA}, which is a non-profit organisation in The United States that believes that every child, regardless of circumstances at birth, can succeed in life. Go Smart is not a website where users can find and sign up for activities, but a model which families, teachers, providers and others can use to find ways to improve the health and development of young children up to the age of five.


Activities that is in Go Smart's model is displayed at their homepage with dynamic loading if the user wants to see more activities. Go Smart offers users to save their favorite activities and filter and search for different activities. The site offers many way to filter out activities, like age, environment, group size and if activities is suitable for non-mobile children or not. The search functionality is not good because the page will refresh every time the user types a letter in to the search field. Therefore users are not able to easily write whole words into the search-field. 

\subsection{Comparison}
Each of the four different solutions is different in their own way. TRDevents is the most similar to skalvi.no with the intention to be a portal that gathers all activity offers in Trondheim and when i comes to functionality. While TRDevents only displays 50 items per page, skalvi.no displays all activities with dynamic loading. TRDevents has sharing-functionality with Facebook, which is not possible on skalvi.no yet. There is no kind of filers or information for users with special needs. The site is therefore not convenient for this project's user group as it is not adapted to their specific needs. 

KUBA and BHSS is most similar to skalvi.no when i comes to who the product is intended for, but the both of them does not have any functionality. Since they don't not have any functionality, it is mush harder to find spesific avtivities on their site than skalvi.no

Go Smart is a very good alternative if you are a provider and want to find different kind of activities you can offer to children. Since Go Smart is a model and skalvi.no i a portal, they do not offer the the same opportunities to the users. Go Samrt and Skalvi.no offers users the opportunity to log in with Facebook.


Functionality that is present on skalvi.no, is family log in, wich is a functionality that non of the others have. Users can create one family-user and create multiple sub-users within the family. With this kind of functionality, a family can log in with one username and password and then choose wich sub-user they want to be. It isalso possible to easily switch between sub-users. With a family-user, parents are able to see what kind of activities their children are attending.


Another functionality that skalvi.no have, is the opportunity to create activities on the site with events that exists on Facebook. When a user logs in with Facebook, the user i able to create activities based on a Facebook-event that the user has been invited to, is interested in or is attending on Facebook. Users can therefore easily and quick create activities on skalvi.no without having to re-enter a lot of information. All the information and pictures in the Facebook-event will automatically be display when a user creates an event.


\section{Development process} \label{s:development_process}
The group chose to utilize scrum \cite{scrum} as the development process during the project. "Scrum is an iterative and incremental agile software development framework for managing product development. \cite{scrumWikipedia}" Because of frequent meetings with the customer and changes to customer requirements, scrum was a suited choice of framework for the project. All group members had previous experience using the scrum methodology from previous projects, which gave the group an advantage of an early start. 

Other frameworks that were taken to consideration was Extreme Programming and Kanban. Both of these frameworks are commonly used in software development, and are both classified as agile processes. These two processes are more indefinite than Scrum, and does not provide all the features Scrum does. The group concluded that with a well documented framework and previous experience, Scrum would be a better choice. Furthermore the customer wanted an agile development process, and required documentation if someone else were to continue the project after the group's completion.


\section{Webserver vs. Webhotel}
The platform for the web portal stood between Webhotel or VPS - virtual private server.  The group had several meetings with the customer regarding this decision. The customer primarily wanted to use web hotel as hosting service and platform. The group proposed to use VPS because of the many advantages using web portal, rather than the constraints with web hotel. 

The group presented to use VPS because it allows for developing the graphical user interface faster and easier by allowing usage of library and frameworks, which web hotels does not.  Thus time and resources would be more efficiently allocated. Security is also higher with the use of VPS because it does not use a shared resource. If the web portal was hosted through a web hotel, the security of the page and its information would only be protect as much as the less secure web page that also uses the same shared resource. Hosting on VPS would provide more security for the users of the web portal, and since the product potentially holds sensitive information about the users, security is not to be taken for granted.  Experience also played a part in the decision. The group has more experience with the usage of VPS compared to web hotels. 

Regarding functionality and the project requirements, the platform could be either one, but the usage of VPS would contribute to an much easier development, and allow usage of the best suitable technology for the given task. 

The group presented the customer with the proposal to use VPS instead of webhotel, and was engaged in a consulting meeting with the customer, outside developer and the group. The group emphasized that the usage of VPS was just a suggestion, and that it was up to the customer to choose. The project could be done with both.


\section{Tools}
\label{tools}
The following section will contain information about all tools used in this project, including tools for project management, communication, document sharing and product development.

\subsection{Project Management and Communication}

\subsubsection{Slack}
The group chose Slack \cite{Slack} as the main communication channel within the group. Slack allows members to create distinct threads, where discussions about different subjects can be discussed. This makes communication easier and organized, compared to other chat-applications like Facebook Messenger. Slack also provides great integration with GitHub, which kept all group members updated on changes in the GitHub repository. 

\subsubsection{Email}
The selected communication channel between the group, the group supervisor and the customer, was email.
Email was chosen because it is easy to maintain; and sender and receiver have their own copies. Also the possibility to send attachments, categorise on subject and has unlimited amount of characters has been a motivation to use email. Both the customer and supervisor preferred email as their main channel to communicate with the group.  

\subsubsection{Toggl}
Toggl \cite{Toggl} was chosen as the group's time tracking tool. It makes it easy to track time, and specify what each group member had been working on. Toggl provides great visualization of time spent on different tasks, with different types of charts and diagrams. It is also open for the entire group, which means every group member can see what other members are working on. Toggl also uses tags so it is easy to divide group tasks. This allows for tracking progress and what type of tasks is most time consuming.  

\subsubsection{Git}
The group chose to use Git \cite{Git} as their version control system. Git is an open source version control system, that allows teams to work on projects with speed and efficiency. Git makes it easy to manage project source code history and replaces Source Code Management (denoted as SCM)  tools, because of cheap local branching, convenient staging areas, and multiple workflows. 

\subsubsection{GitHub}
\label{GitHub}
GitHub\cite{GitHub} is a hosting service for distributed revision- and version control systems. GitHub offers SCM functionality and access control, in addition to many more collaboration and project features. Such as wiki-pages, task management tools, and different types of repositories for your projects such as public and private. The groups decision to use Github was based on customer's demand, that the project was to be open source and hosted on Github. 

\subsubsection{Waffle.io}
\label{Waffle.io}
The group has chosen Waffle.io \cite{Waffle} as the main task manager tool. The entire product backlog and all sprint backlogs, were managed through Waffle.io. Waffle.io has great integration with GitHub, and will automatically synchronize tasks with issues on GitHub. The opportunity to create burndown-charts in relation to the work accomplished in Waffle.io or at GitHub is also possible. This is the main reason for why the group chose to work with Waffle.io compared to other known tools like Trello.

\subsubsection{Google Drive}
For document sharing, the group decided to use Google Drive \cite{GoogleDrive}. Google Drive has a great way to share documents and information between a group. It also allows the group members to write in the same document, and edit the same files in real time. That makes it easy for all members to collaborate. Google Drive was mainly chosen because of the easy collaboration opportunities, which other competitors like Dropbox or OneDrive does not support. Other features that Google Drive supports and offers is the integration with other applications and document types, such as docx, spreadsheets, powerpoints. These document types are essential in a product development, and therefore a natural choice for the group, as of having different file types in different locations were considered inconvenient.

\subsubsection{ShareLaTeX and LaTeX}
ShareLatex \cite{ShareLatex} is an online editor for writing text documents, which allows collaborative work in real time. Latex is a typesetting system, created for writing scientific and technical documents.

The group selected Latex and ShareLatex due to the fact that it is easy to construct and maintain the document, even when the document becomes large. Here Latex are a better choice than for example Google Docs. The group also chose Latex, because of the possibility to store back-ups in GitHub and Dropbox.

\subsubsection{Gantt Project} \label{sss:Gant_Project} 
Gantt Project \cite{Gantt} is a free desktop scheduling and management app, which provided the group a convenient approach to create a Gantt Diagram (see section \ref{GanttDiagram}). 

The group selected to utilise Gantt Project to create the Gantt Diagram because of the many features provided by the app; such as efficiently adding tasks and sub-tasks, set start and finish dates for each specific task. The app supplied the group with a visualisation of the project life cycle based on tasks and dates added.    

\subsubsection{PyCharmIDE}
PyCharm\cite{PyCharm} is a Integrated Development Environment for professional Python developers. The group chose to use PyCharm, because it helped produce code more effectively, and provided good error checking. The entire group also used the professional edition of PyCharm, providing additional support for web development and the Django framework, resulting in an IDE supporting the entire project.

\subsection{Front End}
\label{frontEnd}
\subsubsection{React}
For front end development the group chose to use components defined in React library for Javascript \cite{React} combined with HTML-files created using Django templates. The React library offers easy creation of components that are dynamic, fast rendering, and is user friendly. This resulted in being able to create web pages that are responsive and load fast. Using React, the overall structure of the project also became organized, easy to understand and update. It also allowed much reuse of code, saving time and complexity. 

\subsubsection{Redux}
\label{redux}
To ensure the application behaved consistently the group decided to use Redux. "Redux is a predictable state container for JavaScript apps" \cite{Redux}. By using Redux the group managed to cope with state mutations and asynchronicity, and manage the state of the data in the application. The group utilised Redux with the React components, and it was principally used to filter and serach for the activities presented to the users, and assure consistent state among the application.  

\subsubsection{Bootstrap}
Another tool used for front end development is Bootstrap \cite{Bootstrap}. This is one of the most used HTML, CSS and Javascript frameworks to develop responsive and mobile websites. The group chose to use Bootstrap because it helps writing organized code, saving time writing CSS-code because of the many pre-written styles and components that group can reuse. This also reduces the code complexity and lines of code. 

\subsubsection{Material Design Lite}
\label{mdl}
Material Design Lite \cite{Material_Design_Lite} is a library for designing dynamic websites, and does not rely on any other Javascript frameworks. \textit{"It aims to optimize for cross-device use, gracefully degrade in older browsers, and offer an experience that is immediately accessible."} \cite{Material_Design_Lite}. The group chose to use Material Design Lite because it helps to create a clean and modern design, behaves consistent on different platforms, and provides a rich library of predefined components ready to use.

\subsection{Back End}
\label{backEnd}
\subsubsection{Amazon Web Services - EC2}
Amazon Web Services (denoted as AWS) is a cloud services platform, offering compute power, database storage and much more \cite{AWS}. The group proposed to host the web portal on a Virtual Private Server (denoted as VPS) instead of web hotel to the customer. Based on the benefits and discussions; the customer decided to use VPS as hosting service, and to go with AWS as provider. The customer chose AWS as the VPS provider because they are one of the major providers on the market. A customer requirement was that the production data was safe and would not disappear. AWS, provide this security and thus fulfills the requirement.

The group choose to use Amazon Elastic Compute Cloud (denoted as EC2) \cite{EC2}. It is a web service that gives complete control of the computing resources and reduces time used to create a VPS instance. The decision was also given by the fact that part of the team have prior experience with EC2 technology. 


\subsubsection{Nginx}
The group chose to use Nginx \cite{nginx} as a reverse proxy and load balance between the EC2 instance and the web portal application. The use of reverse proxy can hide the existence of an origin server, and to reduce the stress-load on the web portal. The group decided to use Nginx instead of e.g. Apache2, because the member who was main responsible for the back end had prior experience with Nginx.

\subsubsection{uWSGI}
uWSGI\cite{uWSGI} is a deployment option on servers like Nginx, and aims at full stack developing for building hosting services. WSGI stands for Web Server Gateway Interface. The reason uWSGI was used in this project is because "A traditional web server does not understand or have any way to run Python applications." - \cite{whyUseWSGI} Thus the group decided to use uWSGI because it can communicate with Nginx and the python application. Nginx takes the user request, passes it to uWSGI, which then passes the request to the application through Unix ports and then get the response from the application, gives it to Nginx which sends the response to the user. Thus, by using Nginx and uWSGI, the application gains more speed and faster responses for the users. The group decided to go with uWSGI instead of e.g Gunicorn, because uWSGI is easier to get started with, allows for easier and more configuration and its speed.


\subsubsection{Python}
\label{python}
The group chose to use Python \cite{python} as programming language for the server side of the web portal, because it is a one of the most common programming languages \cite{Programminglanguagespopularity}, thus Python meets the requirement for making the product easy to further develop for a new group. It is a simple programing language, and many know it in advance. Python is a programming language that integrates with many other programming languages like Java with Jython or C with CPython. Python also has a huge community and frameworks, such as Django for web devolpment, which has a lot of features included that the project need and would take a lot of time to implement without it, like administration page and database communication.

\subsubsection{Django}
\label{django}
Django \cite{django} was chosen as the main framework for writing the back end of the web portal. Django is a high-level framework for developing websites with a back end consisting of Python. All group members have experience with Python, and combined with the Django framework, this will allow for rapid development. Furthermore Django is a framework that takes care of a lot of the the backend specific programming, required by all websites. Such as login, database connection and routing. Since the product is a “proof of concept”, these features allows the group to concentrate on the application and rapid development.  

\subsubsection{SQLite}
SQLite\cite{SQLite} was chosen as database type, and used as data store on the web portal. The web portal was developed as “proof of concept”, thus the portal does not need a full size database, where many client programs are sending data to the same database over a network as it would if the web portal was to be developed for production. The project to be developed only needed essential information, not raw table content and a database that would have many client programs with concurrent requests. 

To illustrate what is possible, and to get the database configured fast, SQLite was chosen because it does not return generic SQL and raw table content. SQLite returns high-level and specific data that does only contain the essential information and suits the purpose of the web portal without excessive features. It is reported that SQLite is often faster than a client/server SQL database engine in this scenario. \cite{Server-sideDatabase} 


\cleardoublepage