%===================================== CHAP 2 =================================

\chapter{Prestudies}


\section{Customer input}
During the first meeting with the customer, the group received input about the overall structure of the desired product, as well as the research required. The customer emphasized the fact that the project is a “Proof of Concept”, and that the group had to research utilizing universal design principles in web applications.

The application to be developed is purposed to be used in Trondheim municipality, with the possibility to be scaled up to a bigger user group. Trondheim municipality wishes to easily allow parents and youths to find, participate and master leisure activities. This user base represented a big group of users, and the project was therefore allocated resources to perform workshops and user tests. Such test would be prior for a successful application and project.

Both server development and graphical user interface is necessary for the group to create the web portal the customer wishes. The group had a relative free choice when it came to technology, but customer input had an impact on the final decisions.

\section{Exsisting Solutions}
\label{alternativeSolutions}
The user group of skalvi.no has a limited amount of exsisting solutions to choose from when it comes to finding leisure activity offers. They are not satisfied with the existing solutions, because the activity offers is not easily accessible and there is often lack of detailed information about the activity.  
The chosen solutions are TRDevents, KUBA and Beitostølen Helsesportsenter. 

\subsubsection{TRDevents}
In Trondheim, TRDevents \cite{TRDevents} is the biggest portal to find activities and other events. These can be registered by both organisations and private people. Their focus is to list all events in Trondheim and it is meant to keep everyone updated about upcoming events and activities. This cite is not convenient for this project's user group as it is adapted to their specific needs.

\subsubsection{Kultur for Barn}
A portal that is more aimed towards this project's user group is KUBA \cite{KUBA}. KUBA stands for "Kultur for Barn" (Culture for children), and is Trondheim municipality's cultural program for children in all ages. KUBA strives to be as diverse and as inclusive as possible, such that everyone can participate on their activities.

\subsubsection{Beitostølen Helsesportsenter}
A more national offer is Beitostølen Helsesportsenter (further denoted as BHSS) \cite{BHSS}. BHSS gives people in all ages with physical, psychological and/or cognitive disabilities the opportunity to rehabilitation through activities. The rehabilitation takes place at their center in Beitostølen located in Øystre Slindre municipality. Their main focus is on opportunities rather than limitations.  


\section{Development process}

\section{Tools}
\label{tools}
The following section will contain information about all tools used in this project, including both tools for project management, communication, document sharing and product development.

\subsection{Project Management and Communication}

\subsubsection{Slack}
The group has chosen Slack \cite{Slack} as the main communication channel within the group. Slack allows members to create distinct threads, where discussions about different subjects can be discussed. This makes communication easier and cleaner, compared to other chat-applications like Facebook Messenger. Slack also provide great integration with GitHub, which means Slack will keep group members updated on all changes at the group's GitHub repository. 

\subsubsection{Email}
The selected communication channel between the group - and the group supervisor and the customer was email.
Email was chosen because it is easy to maintain and sender and receiver have their own copies. Also the possibility to send attachments, categorise on subject and has unlimited amount of characters has been a motivation to use email.  

\subsubsection{Toggl}
Toggl \cite{Toggl} is chosen as the group's time tracking tool. It makes it easy to track time, and specify what each group member are working on. Toggl provides great visualization of time spent on different tasks, with different types of charts and diagrams. It is also open for the entire group, which means every group member can see what other members are working on. Toggl also uses tags so it is easy to divide group tasks. This allows for tracking progress and what type of tasks is most time consuming.

\subsubsection{Git}
Git \cite{Git} is a open source, version control system, which allows teams to work on projects that are small or big with speed and efficiency. Git is a tool to manage project source code history and replaces Source Code Management (denoted as SCM)  tools, because of cheap local branching, convenient staging areas, and multiple workflows.

\subsubsection{GitHub}
\label{GitHub}
GitHub\cite{GitHub} is a hosting service for distributed revision- and version control systems. GitHub offers SCM functionality, access control and many more collaboration and project features. Such as wiki-pages, task management tools, and different types of repositories for your projects such as public and private. The groups decision to use Github was based on the fact, that the customer demanded that the project to be open source and hosted on Github. 

\subsubsection{Waffle.io}
\label{Waffle.io}
The group has chosen Waffle.io \cite{Waffle} as the main task manager tool. The entire product backlog and all sprint backlogs, are managed with Waffle.io. Waffle.io has great integration with GitHub, and will automatically syncronize tasks with issues on GitHub. The opportunity to create burndown-charts in relation to the work accomplished in Waffle.io or at GitHub is also possible. This is the main reason for why the group chose to work with Waffle.io compared to other known tools like Trello.

\subsubsection{Google Drive}
For document sharing, the group chose to use Google Drive \cite{GoogleDrive}. Google Drive has a great way to share documents and information between a group. It also allows the group members to write in the same document, and edit the same files in real time. That makes it easy for all members to collaborate. Google Drive was mainly chosen because of the easy collaboration opportunities, which other competitors like Dropbox or OneDrive does not support. 

\subsubsection{ShareLaTeX and LaTeX}
ShareLatex \cite{ShareLatex} is an online editor for writing documents, which allows collaborative work in real time. Latex is a typesetting system, created for writing scientific and technical documents.

The group selected Latex and ShareLatex due to the fact that it is easy to construct and maintain the document, even when the document becomes large. Here Latex are a better choice than for example Google Docs. The group also chose Latex, because of the possibility to store back-ups in GitHub and Dropbox.

\subsubsection{PyCharmIDE}
PyCharm\cite{PyCharm} is a professional Integrated Development Environment for professional Python developers. The group chose to use PyCharm, because it helped produce code more effectively. The entire group also got professional edition version of PyCharm, which makes it the best tool for developing rapid with Python, and has great support for Django. 

\subsection{Front End}
\subsubsection{React}
For front end development the group chose to use components defined in React \cite{React} combined with HTML-files created using Django templates. The React library offers easy creation of components that are dynamic, fast rendering, and is user friendly. This resulted in being able to create web pages that are responsive and load fast. Using React, the overall structure of the project also became organized, easy to understand and update. It also allowed much reuse of code, which saved time and complexity. 

\subsubsection{Redux}
\label{redux}
To ensure the application behaved consistently the group decided to use Redux. "Redux is a predictable state container for JavaScript apps" \cite{Redux}. By using Redux the group managed to cope with mutations and asynchronicity, and manage the state of the data in the application. The group utilised Redux with the React components, and it was principally used to filter the activities presented to the users.  

\subsubsection{Bootstrap}
Another tool used for front end development is Bootstrap \cite{Bootstrap}. This is one of the most used HTML, CSS and Javascript frameworks to develop responsive and mobile websites. The group chose to use Bootstrap because it helps writing cleaner code and therefore the group  saves time writing CSS-code. 

\subsection{Back End}
\subsubsection{Amazon Web Services - EC2}
Amazon Web Services (AWS) is a cloud services platform, offering compute power, database storage and much more \cite{AWS}. The group proposed to host the web portal on a Virtual Private Server (further denoted as VPS) instead of web hotel to the customer. Based on the benefits and discussions; the customer decided to use VPS as hosting service, and to go with Amazon Web Services as provider. Amazon was chosen as a provider based on the fact that they are big in the game and because they can "survive" anything, which is an important requirement for the customer. That the web portal and it's data, will not disappear over night. 

The group choose to use Amazon Elastic Compute Cloud (further denoted as EC2) \cite{EC2}. It is a web service that gives complete control of the computing resources and reduces time used to create a VPS instance. The decision was also given by the fact that part of the team have prior experience with EC2 technology. 


\subsubsection{Nginx}
The group chose to use Nginx \cite{nginx} as a reverse proxy and load balance between the EC2 instance and the web portal application. The use of reverse proxy can hide the existence of an origin server, and to reduce the stress-load on the web portal. The group decided to use Nginx instead of e.g. Apache2, because the group lead of back end had prior experience with Nginx.


\subsubsection{Python}
The group chose to use Python as the programming language for the server side of the web portal. This was a decision based on the fact that \textit{"Python is a programming language that lets you work quickly and integrate systems more effectively"} \cite{python}. The group has previous experience with the language and it is a widely spread language. So when the project is done, and someone else is to further develop the project, they can easily get started without the need to learn a new language.

\subsubsection{Django}
\label{django}
Django \cite{django} was chosen as the main framework for writing the entire back end of the web portal. Django is a high-level framework for developing websites with a back end consisting of Python. All group members have experience with Python, and combined with the Django framework, this will allow for rapid development.  

\subsubsection{SQLite}
SQLite\cite{SQLite} was chosen as the type of database. SQLite uses normal SQL queries, which the entire group already had knowledge about. SQLite is a light weight version of MySQL, which would be more than enough to satisfy the customer's demands. Django also provides great support for SQLite, which makes it easier for the group to use SQLite during development and in production.


\cleardoublepage