%===================================== CHAP 6 =================================

\chapter{Testing}

\section{Test strategy}
The goal is to deliver a product with high quality and that works according to the given requirements. Therefor the group aims to test most parts of the product using acceptance testing. Acceptance testing is testing the requirements and that the product satisfies the users needs \cite{acceptanceTesting}. The groups chose this testing method, because this was the only testing method the customer wanted the group to spend time on.

\subsection{Test plan}
The group decided to create a test plan, see table \ref{Test plan}  to get an overview of what types of acceptance tests were to be performed during the project and to what time.

The customer wanted to include the users as much as possible during the development process. This was done to get feedback on the usability and the functionality. Therefore the test plan includes the acceptance tests. The acceptance tests, \ref{Test cases}, are based on the user cases, see chapter \ref{User cases}.

\begin{longtable}{|l|l|}
\hline
\rowcolor{Gray}
\textbf{Dates} & \textbf{Test cases to be tested} \\
\hline
13.03 - 17.03 (Sprint 3) & UCXX, UCXX,   \\
\hline
20.03 - 24.03 (Sprint 3) & Workshop with users, free testing\\
\hline
24.04 (Sprint 5) & All test cases \\
\hline
15.05 - 19.05 (Sprint 7) & Final test, test all cases  \\
\hline
\caption{Test plan}
\label{Test plan}
\end{longtable}


\subsection{Focus Group}

\subsubsection{Children And Parents}

\subsubsection{Activity Providers And System Maintainers}


\subsection{Acceptance Test}
\subsubsection{Test cases}
\label{Test cases}


\section{Test Execution}
\subsection{Focus Group}

\subsection{Acceptance Test}
\subsubsection{Test cases}


\cleardoublepage